%! Suppress = Makeatletter
%! Suppress = Makeatletter

\documentclass[a4paper,14pt]{extarticle}

\usepackage[utf8]{inputenc}
\usepackage[T1]{fontenc}
\usepackage[russian]{babel}
\usepackage[top=2cm, bottom=2cm, left=2cm, right=2cm]{geometry}
\usepackage{graphicx}
\usepackage{amsmath,mathtools}
\usepackage{amssymb}
\usepackage{listings}
\usepackage{xcolor}
\usepackage{booktabs}
\usepackage{lipsum}
\usepackage{algorithm}
\usepackage{algpseudocode}
\usepackage{titlesec}
\usepackage{multirow}
\usepackage{svg}
\usepackage{tikz}

\makeatletter
\renewcommand{\ALG@name}{Алгоритм}
\makeatother

\titleformat{\section}
{\normalfont\fontsize{16}{18}\bfseries}{\thesection}{1em}{}

\lstset{ %
    extendedchars=\true,
    keepspaces=true,
    language=C++,		    	    %choose the language of the code
    basicstyle=\small,		% the size of the fonts that are used for the code
    numbers=left,					% where to put the line-numbers
    numberstyle=\footnotesize,	    % the size of the fonts that are used for the line-numbers
    stepnumber=1,					% the step between two line-numbers. If it is 1 each line will be numbered
    numbersep=10pt,					% how far the line-numbers are from the code
    backgroundcolor=\color{white},	% choose the background color. You must add \usepackage{color}
    showspaces=false, 		    	% show spaces adding particular underscores
    showstringspaces=false,			% underline spaces within strings
    showtabs=false,					% show tabs within strings adding particular underscores
    frame=tb,                  		% adds a frame around the code
    tabsize=4,						% sets default tabsize to 2 spaces
    captionpos=b,					% sets the caption-position to bottom
    breaklines=true,				% sets automatic line breaking
    breakatwhitespace=false,		% sets if automatic breaks should only happen at whitespace
    escapeinside={\%*}{*)},			% if you want to add a comment within your code
    morekeywords={pragma, parallel, omp, collaps, schedule, shared, reduction, collapse},
    xleftmargin=1.5em,
    framexleftmargin=2em
}

\begin{document}

    \begin{titlepage}
        \centering
        Санкт-Петербургский политехнический университет Петра Великого \\
        \vspace{0.15cm}
        Высшая школа интеллектуальных систем и суперкомпьютерных технологий \\
        \vspace{6.5cm}

        {
            \centering \textbf{Лабораторная работа} \\
            \vspace{0.15cm}
            \textbf{Дисциплина}: Параллельные вычисления \\
            \vspace{0.15cm}
            \textbf{Тема}: Расчет определителя матрицы.
        } \\

        \vspace{6.5cm}

        \begin{table}[H]
            \begin{tabular}{p{\textwidth}@{}r}
            {Выполнил студент гр. 3540901/91502} \hfill { %! Suppress = LineBreak
                $\underset{\text{(подпись)}}{\underline{\hspace{0.25\textwidth}}}$ Ю. И. Леженин
            } \\
            {Преподаватель} \hfill { %! Suppress = LineBreak
                $\underset{\text{(подпись)}}{\underline{\hspace{0.25\textwidth}}}$ И. В. Стручков ~
            } \\ ~ \\
            {} \hfill { "\underline{\hspace{0.08\textwidth}}" \underline{\hspace{0.2\textwidth}}2020 г.} \\
            \end{tabular}
        \end{table}

        \vfill

        {\centering Санкт-Петербург \\ \vspace{0.15cm} 2020 }

    \end{titlepage}

    \tableofcontents

    \newpage
    \section{Алгоритм вычисления определителя матрицы.}

    Общепринятый на практике метод вычисления определителя матрицы основан на выполнении $LU$-разложения.
    Квадратная матрица $A$ может быть представлена в виде:
    \begin{equation*}
        A = P^{-1} \, L \, U,
    \end{equation*}
    где $P^{-1}$ -- матрица перестановок, $L$ -  нижняя унитреугольная матрица, а $U$ верхняя треугольная матрица.
    В этом случае
    \begin{equation*}
        \det(A) = \det(P^{-1}) \cdot \det(L) \cdot \det(U) = \epsilon \cdot 1 \cdot \det(U),
    \end{equation*}
    где $\epsilon \in \{-1, +1\}$ -- знак матрицы перестановок:
    -1 для нечетного числа перестановок и +1 для четного числа перестановок.
    Определитель треугольной матрицы $U$ равен произведению диагональных элементов.

    Поиск треугольной матрицы $U$ может быть выполнен с помощью метода Гаусса.
    Для этого необходимо каждую строку $i = \overline{1, n}$ исходной матрицы $A$ вычесть из каждой нижней строки
    $j = \overline{i+1, n}$, домножив на коэффициента $r = a_{ij}/a_{ii}$.
    При вычислении коэффициента $r$ может возникнуть деление 0,
    поэтому необходимо предварительно выполнить перестановку строк так, чтобы $a_{ii} \neq 0, \: i = \overline{1, n}$.
    Если перестановку выполнить не возможно, то матрица $A$ является сингулярной и $\det(A)$ = 0.
    Описанная процедура представлена в виде алгоритма~\ref{alg:det}.
    Данный алгоритм имеет временную сложность $O(n^3)$.

    \begin{algorithm}[h]
        \caption{Вычисление определителя матрицы.}
        \label{alg:det}
        \begin{algorithmic}[1]
            \Function{determinant}{$A, \; n$}%\Comment Матрица $M$ размером $n \times n$
            \State $p$, singular $\gets$ swapRows($A, \; n$)\Comment $p$ -- число перестановок
            \If {singualr}
            \State \textbf{return} $0$
            \EndIf
            \For {$i \gets  \overline{1, n}$} \Comment Итерация про строкам
            \For {$j \gets \overline{i+1, n}$} \Comment Итерация по строкам ниже
            \State $r \gets a_{ij} / a_{ii}$
            \For {$k \gets  \overline{1, n}$} \Comment Вычитание домноженной строки
            \State $a_{ik} \gets a_{jk} - r \cdot a_{ik} $
            \EndFor
            \EndFor
            \EndFor
            \State $d \gets 1$
            \For {$i \gets \overline{1, n}$}\Comment Перемножение диагональных элементов
            \State $d \gets d \cdot a_{ii}$
            \EndFor
            \algstore{det}
        \end{algorithmic}
    \end{algorithm}

    \begin{algorithm}[h]
        \begin{algorithmic}[1]
            \algrestore{det}
            \If {$p \mod 2 = 1$}
            \State $d \gets -1 \cdot d$
            \EndIf
            \State \textbf{return} $d$
            \EndFunction
        \end{algorithmic}
    \end{algorithm}

    \section{Реализация вычисления определителя матрицы.}

    Описанный выше алгоритм вычисления определителя матрицы был реализован на языке C++ с использованием
    стандарта OpenMP для распараллеливания вычислений.
    Соответствующий фрагмент исходного кода представлен в листинге~\ref{lst:det}.
    \vspace{0.5em}
    \begin{lstlisting}[
        caption={Реализация алгоритма вычисления определителя матрицы.},
        label={lst:det}
    ]
const auto[singular, permutations] = swapRows(matrix);
if (singular) {
    return 0.0;
}

const Index n = matrix.columnNumber();
std::vector<T> ratio(n); ratio.resize(n);
T determinant = permutations % 2 == 0 ? 1 : -1;

#pragma omp parallel default(none) shared(matrix, ratio)
reduction(*:determinant) if (enable_parallel)
{
    for (Index i = 0; i < n; ++i) {                 // loop 1
        #pragma omp for schedule(static)
        for (Index j = i + 1; j < n; ++j) {         // loop 2
            ratio[j] = matrix(j, i) / matrix(i, i);
        }
        #pragma omp for collapse(2) schedule(static)
        for (Index j = i + 1; j < n; ++j) {         // loop 3
            for (Index k = 0; k < n; ++k) {         // loop 4
                matrix(j, k) =
                    matrix(j, k) - ratio[j] * matrix(i, k);
            }
        }
    }
    #pragma omp for schedule(static)
    for (Index i = 0; i < n; ++i) {                 // loop 5
        determinant *= matrix(i, i);
    }
}

return determinant;
\end{lstlisting}

    Распараллеливание вычислений достигалось за счет параллельного выполнения циклов.
    Всего для вычисления определителя в программе выполняется пять циклов,
    в которых выполняется работа со следующими данными: матрица matrix,
    вектор коэффициентов ratio и значение определителя determinant.
    Все пять циклов объединены в одну параллельную секцию, чтобы избежать многократного создания и уничтожения потоков.
    Первый внешний цикл \textit{loop1} выполняется последовательно, так как его итерации зависимы,
    остальные циклы выполняются параллельно.
    В данном случае исключена возможность одновременного обращения к одной ячейке матрицы или вектора коэффициентов из
    разных потоков для выполнения записи, поэтому переменные matrix и ratio разделяются между всеми потоками с помощью
    директивы shared.
    Конечное значение определителя получается путем перемножения итоговых значений переменной determinant в каждом потоке,
    что указывается с помощью директивы reduction.
    Также циклы \textit{loop3} и \textit{loop4} были объединены с помощью директивы collapse,
    чтобы уменьшить объем вычислений в рамках одной итерации.

    Для созданной программы был разработан набор тестов с помощью библиотеки
    Google Test\footnote{https://github.com/google/googletest}, в которых проверялось
    корректность вычисления определителя случайной матрицы при параллельном и последовательном исполнении программы.

    \section{Время выполнения программ}

    Для созданной программы было измерено время выполнения при разном числе рабочих потоков и разном размере матрицы
    (под размером матрицы понимается число строк и столбцов квадратной матрицы).
    Измерения выполнялись с помощью библиотеки Google Benchmark\footnote{https://github.com/google/benchmark} на компьютере,
    параметры которого представлены в таблице~\ref{tab:par}.
    \begin{table}[b]
        \centering
        \begin{tabular}{lr}
            \toprule
            Процессор &  Intel\textregistered{} Core\texttrademark{}  i7-8550U  \\
            Тактовая частота & 4000 МГц \\
            \midrule
            Количество реальных CPU & 4 \\
            Количество виртуальных CPU & 8 \\
            \midrule
            Объема кеша данных L1 & 4 $\times$ $\;\,\;\,$32 КБ  \\
            Объема кеша инструкций L1 & 4 $\times$ $\;\,\;\,$32 КБ  \\
            Объема общего кеша L2 & 4 $\times$ $\;\,$256 КБ \\
            Объема общего кеша L3 & 1 $\times$ 8192 КБ \\
            \bottomrule
        \end{tabular}
        \caption{Параметры компьютера, на котором проводились измерения времени выполнения программы.}
        \label{tab:par}
    \end{table}
    С заданным количеством потоков и размером матрицы программа запускалась 100 раз.
    Среднее значение, медиана и среднеквадратическое отклонение (СКО) времени выполнения программы для каждой
    конфигурации приведены в таблице~\ref{tab:res}.
    \begin{table}[t]
        \centering
        \begin{tabular}{ccrrrr}
            \toprule
            Размер матрицы & Число потоков & Среднее & Медиана & СКО  \\
            \midrule
            \multirow{4}{*}{128} & 1 & 1.5549 & 1.5528 & 0.0544 \\
            & 2 & 0.8697 & 0.8601 & 0.0330 \\
            & 4 & 0.6460 & 0.6447 & 0.0047 \\
            & 8 & 0.7059 & 0.6983 & 0.0491 \\
            \midrule
            \multirow{4}{*}{256} & 1 & 11.5983 & 11.5735 & 0.0741 \\
            & 2 & 5.8774 & 5.8610 & 0.0340 \\
            & 4 & 3.8964 & 3.8874 & 0.0424 \\
            & 8 & 4.0800 & 4.0222 & 0.1638 \\
            \midrule
            \multirow{4}{*}{512} & 1 & 90.6609 & 90.3180 & 0.8492 \\
            & 2 & 46.0939 & 45.9268 & 0.3681 \\
            & 4 & 29.6951 & 29.1847 & 2.1502 \\
            & 8 & 29.7776 & 29.5715 & 0.6992 \\
            \midrule
            \multirow{4}{*}{1024} & 1 & 726.3535 & 723.7002 & 7.5450 \\
            & 2 & 379.3393 & 378.1386 & 4.8630  \\
            & 4 & 254.5211 & 246.6211 & 19.7006 \\
            & 8 & 242.0866 & 240.3842 & 5.3130 \\
            \midrule
            \multirow{4}{*}{2048} & 1 & 6362.3304 & 6330.5451 & 59.5313 \\
            & 2 & 3393.8876 & 3380.3598 & 50.7967 \\
            & 4 & 2459.6841 & 2412.11952 & 107.47609 \\
            & 8 & 2388.7951 & 2378.5972 & 30.4998 \\
            \bottomrule
        \end{tabular}
        \caption{Результаты измерений времени выполнения программы.}
        \label{tab:res}
    \end{table}
    На рисунке~\ref{fig:res} представлена зависимость среднего времени выполнения программы от размера матрицы
    при разном количестве потоков.
    Дополнительно, в таблице~\ref{tab:resm} показан коэффициент ускорения:
    во сколько раз удалось уменьшить время выполнения программы, увеличив число потоков.
    \begin{table}[h!]
        \centering
        \begin{tabular}{cccccc}
            \toprule
            \multirow{2}{*}{Число потоков} & \multicolumn{5}{c}{Размер матрицы} \\
            \cmidrule(l){2-6}
            &  128 & 256 & 512 & 1024 & 2048 \\
            \midrule
            1 & 1.0000 & 1.0000 & 1.0000 & 1.0000 & 1.0000 \\
            2 & 1.7879 & 1.9734 & 1.9669 & 1.9148 & 1.8746 \\
            4 & 2.4069 & 2.9767 & 3.1065 & 2.8538 & 2.5866 \\
            8 & 2.2027 & 2.8427 & 3.0658 & 3.0004 & 2.6634 \\
            \bottomrule
        \end{tabular}
        \caption{Коэффициент ускорения выполнения программы.}
        \label{tab:resm}
    \end{table}

    \begin{figure}[p]
        \centering
        \includesvg[scale=1]{../output/plot.svg}
        \includesvg[scale=1]{../output/plot_log.svg}
        \caption{
            Зависимость среднего времени выполнения программы от размера матрицы при разном количестве потоков
            в линейном (сверху) и логарифмическом (снизу) масштабах.
        }
        \label{fig:res}
    \end{figure}

    \section{Выводы.}

    В ходе выполнения работы была разработана программа, которая выполняет вычисление определителя матрицы.
    Вычисления были распараллелены с помощью директив OpenMP\@.
    Время выполения программы было измерено в зависимости от размера матрицы и числа рабочих потоков.

    В лучшем случае, для матрцы размером 512,
    удалось добиться уменьшения времени выполнения в 1.9669 раз при 2 потоках, в 3.1065 раз при 4 потоках и в 3.0658 при 8 потоках.
    При 4 и 8 потоках программа всегда выполняется приблизительно за одно и то же время.
    Это объясняется тем, что процессор компьютера, на котором проводились измерения,
    имеет 4 реальных вычислительных ядра.
    Для программ, которые интенсивно используют процессор и совершают мало операций ввода/вывода,
    дальнейшее увелечение числа потоков, не позволяет уеньшить время выполнения.


\end{document}
